%example of a SWIG
%%%%%%%%%%%%%%%%%%%%%%%%%%%%%%%%%%%%%%%%%%%%%%%%%%%%%%%%%%%%%%%%%%%%%%%%

\documentclass{article}
\usepackage{verbatim}
\usepackage{tikz}
%nice font
\usepackage[default]{sourcesanspro}
\usepackage[T1]{fontenc}
\usepackage{bm}
\usepackage{color}

\usepackage[paperwidth=6in,paperheight=3in,left=.3in,top=.3in,right=.3in,bottom=.3in]{geometry}

%list of libraries at http://tex.stackexchange.com/questions/42611/list-of-available-tikz-libraries-with-a-short-introduction
\usetikzlibrary{mindmap,backgrounds, snakes, shapes, positioning}
\newcommand{\seq}{\,{=}\,} % special equals (

\newcommand{\mndim}{0.9cm}
\newcommand{\ndist}{1.0cm}
\newcommand{\ndistb}{0.9cm}

%custom node style
\tikzstyle{leftsplit} = [semicircle, shape border rotate=90, anchor=east,           inner xsep=1pt,  minimum height=\mndim, minimum width=\mndim, very thick, draw]
\tikzstyle{rightsplit} = [semicircle, shape border rotate=270, anchor=west,      inner xsep=1pt,  minimum height=\mndim, minimum width=\mndim, very thick, draw]
\tikzstyle{topsplit} = [semicircle, shape border rotate=0, anchor=south,          inner ysep=1pt,  minimum height=\mndim, minimum width=\mndim, very thick, draw]
\tikzstyle{bottomsplit} = [semicircle, shape border rotate=180, anchor=north, inner ysep=1pt,  minimum height=\mndim, minimum width=\mndim, very thick, draw]

\tikzstyle{swignode} = [ellipse, inner sep=1pt,  minimum height=\mndim, minimum width=\mndim, very thick, draw]
\tikzstyle{swigbox} = [rectangle, inner sep=1pt,  minimum height=\mndim, minimum width=\mndim, very thick, draw]


\begin{document}
\begin{tikzpicture}[node distance=\ndist, very thick]
     \node[leftsplit] (X) {$\bm{X}$};
     \node[red, rightsplit, right = of X, xshift=-\ndistb] (x) {$\bm{ x}$};
     \node[swignode, above right= of x](z){$\bm{ Z }$};
     \node[swignode, below right = of z](y){$\bm{ Y({\color{red} x_1}) }$};
    
   \path  [latex-,shorten <= .05cm, shorten >= .05cm] (z) edge[] (X); %-latex gives arrow head shape
   \path  [-stealth,shorten <= .05cm, shorten >= .05cm] (z) edge (y); %stealth gives arrow head shape (others: angle 60, stealth', latex, latex')
   \path  [-latex,shorten <= .05cm, shorten >= .05cm] (x) edge (y);
   \path  [-latex] (u) edge[out=0, in=-90] (y2);    ;
  

\end{tikzpicture}

\end{document}